\documentclass[../../main.tex]{subfiles}
\graphicspath{{img/}}
\begin{document}

\section{围棋}

\begin{minipage}[t]{.7\textwidth}

    \paragraph*{游戏人数} $2$人

    \paragraph*{游戏道具} $19\times19$的棋盘,黑色和白色棋子

    \paragraph*{游戏目标} 在棋盘上争夺地盘,终局时地盘(包含棋子本身占据的点)更多的一方获胜

    \paragraph*{游戏过程}

    \begin{enumerate}
        \item $2$名玩家中,一名玩家下黑棋,另一名玩家下白棋。
        \item 从下黑棋的玩家开始,双方轮流落子。落子必须落在棋盘的交叉点上。棋子落下后不能移动。
        \item 一片棋子指的是若干个同色且上下左右相邻的棋子。
        \item 一片棋子的气指的是和这些棋子上下左右相邻的空交叉点。
        \item 当一片棋子被对方棋子完全包围(没有气)时,这些棋子从棋盘上被提走。
        \item 若双方都有没有气的棋子,优先提走对方的棋子。不可以出现无法提走对方的棋子,但自己的棋子没有气的情况。
        \item 不允许出现局面的循环。(例如打劫)
        \item 围棋中允许停一手(即不在任何位置下棋)。当双方连续停了两手之后,游戏立刻结束。
        \item 当双方都认为没有需要下棋的地方时,游戏结束。双方计算地盘,地盘多的一方胜。
    \end{enumerate}

    \paragraph*{贴目规则} 为平衡先手优势,一般会给后手加上一定数量的地盘(中国规则是$7.5$目,日韩规则是$6.5$目)。小数贴目是为了防止平局。

    \paragraph*{基本策略}

    \begin{itemize}
        \item 金角银边草肚皮:从角落开始围地最有效率,其次是边,最后才是中央。
        \item 避免愚型:不要把大量棋子聚成一团。
        \item 保持棋子连接:将棋子连接在一起可以增强生存能力,避免被分断攻击。
        \item 做眼求活:一片棋需要至少两个真眼才能确保不被提走,这是求活的关键。
        \item 打入与侵消:当对方围起大模样时,可以考虑打入破坏其地盘或从外部侵消。
        \item 厚势与实地:厚势指外势强大但暂时没有确定地盘的棋子,可以用来攻击或后续围地;实地指已经确定的稳固地盘。
    \end{itemize}
\end{minipage}
\begin{minipage}[t]{.3\textwidth}
    \begin{figure}[H]
        \centering
        \begin{tikzpicture}[scale=0.2]
            \goBoard
        \end{tikzpicture}
        \caption{标准的围棋棋盘}
    \end{figure}
    \begin{figure}[H]
        \centering
        \begin{tikzpicture}[scale=0.2]
            \goBoard

            \goLikeBlackStone{5}{4}
            \goLikeBlackStone{5}{6}
            \goLikeBlackStone{4}{5}
            \goLikeBlackStone{6}{5}
            \goLikeWhiteStone{5}{5}

            \goLikeBlackStone{16}{0}
            \goLikeBlackStone{17}{1}
            \goLikeBlackStone{18}{0}
            \goLikeWhiteStone{17}{0}

            \goLikeBlackStone{0}{17}
            \goLikeBlackStone{1}{18}
            \goLikeWhiteStone{0}{18}

            \goLikeBlackStone{15}{14}
            \goLikeBlackStone{15}{16}
            \goLikeBlackStone{16}{14}
            \goLikeBlackStone{16}{16}
            \goLikeBlackStone{14}{15}
            \goLikeBlackStone{17}{15}
            \goLikeWhiteStone{15}{15}
            \goLikeWhiteStone{16}{15}
        \end{tikzpicture}
        \caption{提子的例子:图中所有白棋都被提走}
    \end{figure}
    \begin{figure}[H]
        \centering
        \begin{tikzpicture}[scale=0.2]
            \goBoard

            \goLikeBlackStone{0}{0}
            \goLikeBlackStone{0}{2}
            \goLikeBlackStone{1}{1}
            \goLikeBlackStone{1}{2}
            \goLikeBlackStone{2}{0}
            \goLikeBlackStone{2}{1}

            \goLikeWhiteStone{9}{11}
            \goLikeWhiteStone{9}{12}
            \goLikeWhiteStone{10}{10}
            \goLikeWhiteStone{10}{12}
            \goLikeWhiteStone{11}{10}
            \goLikeWhiteStone{11}{11}
            \goLikeWhiteStone{11}{12}
            \goLikeWhiteStone{12}{10}
            \goLikeWhiteStone{12}{12}
            \goLikeWhiteStone{13}{10}
            \goLikeWhiteStone{13}{11}
            \goLikeWhiteStone{13}{12}
        \end{tikzpicture}
        \caption{眼的例子:有两个真眼的棋子永远不会被提}
    \end{figure}
    \begin{figure}[H]
        \centering
        \begin{tikzpicture}[scale=0.2]
            \goBoard

            \goLikeBlackStone{2}{4}
            \goLikeBlackStone{3}{3}
            \goLikeBlackStone{3}{5}
            \goLikeBlackStone{4}{4}

            \goLikeWhiteStone{4}{3}
            \goLikeWhiteStone{4}{5}
            \goLikeWhiteStone{5}{4}
        \end{tikzpicture}
        \caption{打劫的例子:白棋提黑棋后,黑棋必须在先别处下一步,然后再提回来}
    \end{figure}
\end{minipage}

\end{document}