\documentclass[../../main.tex]{subfiles}
\graphicspath{{img/}}
\begin{document}

\section{围点棋}

\begin{minipage}[t]{.7\textwidth}

    \paragraph*{游戏人数} $2$人

    \paragraph*{游戏道具} 一定大小的方格纸,和两种不同颜色的笔

    \paragraph*{游戏目标} 用自己的点尽可能围住对手的点

    \paragraph*{游戏过程}

    \begin{enumerate}
        \item $2$名玩家分别使用不同颜色的笔。
        \item 双方轮流画点。点必须画在方格纸的交叉点上,不能在已经有点的位置画点。
        \item 围点棋一般会使用固定开局,如右图所示。
        \item 轮到一名玩家行动时,如果若干个自己的点组成一个环,其中相邻的点横竖或对角相邻(共$8$个方向),且环中至少包含一个对手的点,那么这名玩家可以画出这个环,并用自己的笔将环内部的区域打上阴影。环不能包含已经打上阴影的区域。
        \item 以上操作可以在画点前后进行任意次。
        \item 不能在阴影内画点。
        \item 允许停一手(即不在任何位置下棋)。当双方连续停了两手之后,游戏立刻结束。
        \item 当双方都认为没有需要下棋的地方时,游戏结束。双方计算各自的围子数量。
    \end{enumerate}

    \paragraph*{贴目规则} 为平衡先手优势,一般会给后手加上一定数量的围子(具体数值可以根据玩家的经验确定)。为了防止平局,建议贴目为小数(例如$0.5$)。
\end{minipage}
\begin{minipage}[t]{.3\textwidth}
    \begin{figure}[H]
        \centering
        \includegraphics[width=.65\textwidth]{dots_opening.png}
        \caption{围点棋的一种开局}
    \end{figure}
    \begin{figure}[H]
        \centering
        \includegraphics[width=.65\textwidth]{dots.png}
        \caption{围点棋的中局。红方和蓝方分别围了$2$和$13$个子,所以蓝方大优}
    \end{figure}
\end{minipage}

\end{document}