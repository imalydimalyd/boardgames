\documentclass[../../main.tex]{subfiles}
\graphicspath{{img/}}
\begin{document}

\section{十字路口棋}

\begin{minipage}[t]{.7\textwidth}

    \paragraph*{游戏人数} $2$人

    \paragraph*{游戏道具} $19\times19$的棋盘,黑色和白色棋子

    \paragraph*{游戏目标} 黑棋的目标是连接棋盘的上下两边,白棋则是左右两边

    \paragraph*{游戏作者} Mark Steere

    \paragraph*{游戏过程}

    \begin{enumerate}
        \item $2$名玩家中,一名玩家下黑棋,另一名玩家下白棋。
        \item 从下黑棋的玩家开始,双方轮流落子。落子必须落在棋盘的交叉点上。棋子落下后不能移动。
        \item 落子的位置只有一条限制:不可以出现右图所示的两种$2\times2$图案。
        \item 如果两枚同色的棋子横竖或对角相邻(共$8$个方向),那么这两枚棋子就是相连的。
        \item 如果黑棋连接了棋盘的上下两边,那么黑棋获胜。
        \item 如果白棋连接了棋盘的左右两边,那么白棋获胜。
        \item 如果一名玩家没有交叉点可以下棋,那么另一名玩家可以一直下棋,直到另一名玩家也没有交叉点可以下棋。(可以证明,只要两名玩家都还没有获胜,就一定有至少一名玩家可以下棋。也就是说,十字路口棋不存在平局。)
    \end{enumerate}

    \pierule

    \paragraph*{基本策略}

    \begin{itemize}
        \item 开局时尽快抢占关键点位(如中心)。
        \item 不要把大量棋子聚成一团,而是先下出线路的大概模样,然后再完成整条线路。
        \item 因为十字路口棋不存在平局,所以只要“阻止对方连成路线”和“自己连成路线”其实是一回事。
        \item 若采用$N$手交换规则,第二名玩家一定会在判断形势之后,选择更有利的一方继续下棋。所以为了防止第二名玩家得利,第一名玩家需要摆出一个尽可能均势的开局。
    \end{itemize}
\end{minipage}
\begin{minipage}[t]{.3\textwidth}
    \begin{figure}[H]
        \centering
        \begin{tikzpicture}[scale=0.2]
            \goBoard
        \end{tikzpicture}
        \caption{十字路口棋棋盘}
    \end{figure}
    \begin{figure}[H]
        \centering
        \begin{tikzpicture}[scale=0.8]
            \goLikeBoard{5}{5}

            \goLikeBlackStone{0}{3}
            \goLikeBlackStone{1}{2}
            \goLikeWhiteStone{0}{2}
            \goLikeWhiteStone{1}{3}

            \goLikeBlackStone{3}{1}
            \goLikeBlackStone{4}{2}
            \goLikeWhiteStone{3}{2}
            \goLikeWhiteStone{4}{1}
        \end{tikzpicture}
        \caption{不可以出现的两种图案}
    \end{figure}
    \begin{figure}[H]
        \centering
        \begin{tikzpicture}[scale=0.8]
            \goLikeBoard{5}{5}

            \goLikeBlackStone{0}{1}
            \goLikeBlackStone{0}{2}
            \goLikeBlackStone{1}{2}
            \goLikeBlackStone{2}{0}
            \goLikeBlackStone{2}{1}
            \goLikeBlackStone{2}{2}
            \goLikeBlackStone{2}{4}
            \goLikeBlackStone{3}{3}
            \goLikeBlackStone{3}{4}
            \goLikeBlackStone{4}{1}
            \goLikeBlackStone{4}{4}

            \goLikeWhiteStone{0}{0}
            \goLikeWhiteStone{0}{3}
            \goLikeWhiteStone{0}{4}
            \goLikeWhiteStone{1}{1}
            \goLikeWhiteStone{1}{4}
            \goLikeWhiteStone{2}{3}
            \goLikeWhiteStone{3}{1}
            \goLikeWhiteStone{4}{0}
            \goLikeWhiteStone{4}{2}
            \goLikeWhiteStone{4}{3}
        \end{tikzpicture}
        \caption{黑棋获胜}
    \end{figure}
    \begin{figure}[H]
        \centering
        \begin{tikzpicture}[scale=0.8]
            \goLikeBoard{5}{5}

            \goLikeBlackStone{1}{0}
            \goLikeBlackStone{1}{2}
            \goLikeBlackStone{1}{4}
            \goLikeBlackStone{2}{1}
            \goLikeBlackStone{2}{4}
            \goLikeBlackStone{4}{1}
            \goLikeBlackStone{4}{3}

            \goLikeWhiteStone{0}{4}
            \goLikeWhiteStone{1}{3}
            \goLikeWhiteStone{2}{2}
            \goLikeWhiteStone{2}{3}
            \goLikeWhiteStone{3}{0}
            \goLikeWhiteStone{3}{3}
            \goLikeWhiteStone{4}{2}
        \end{tikzpicture}
        \caption{白棋获胜}
    \end{figure}
\end{minipage}

\end{document}