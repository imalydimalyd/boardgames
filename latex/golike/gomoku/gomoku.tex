\documentclass[../../main.tex]{subfiles}
\graphicspath{{img/}}
\begin{document}

\section{五子棋}

\begin{minipage}[t]{.7\textwidth}

    \paragraph*{游戏人数} $2$人

    \paragraph*{游戏道具} $15\times15$的棋盘(也可以用$19\times19$等尺寸代替)和棋子

    \paragraph*{游戏目标} 抢先将自己的棋子形成五连

    \paragraph*{游戏过程}

    \begin{enumerate}
        \item $2$名玩家中,一名玩家下黑棋,另一名玩家下白棋。
        \item 从下黑棋的玩家开始,双方轮流落子。落子必须落在棋盘的交叉点上。除此之外,棋子的落点没有任何限制。
        \item 玩家落子完毕之后,如果自己的棋子形成了五连(五枚自己颜色的棋子在一条直线或斜线上相邻成一排),则这名玩家立即获胜。
        \item 若棋盘下满后仍然没有玩家形成五连,那么游戏为平局。
    \end{enumerate}

    \pierule

    \paragraph*{基本策略}

    \begin{itemize}
        \item 当对手形成冲四(下一步就能形成五连的四枚棋子)之后,为了防止对手形成五连,必须堵住对手的冲四(除非自己可以直接形成五连)。
        \item 当一名玩家形成活四(四枚自己颜色的棋子在一条直线或斜线上相邻成一排,且两端没有对手的棋子阻挡)之后,无论对手堵在那一边,都可以下在另一边形成五连。所以活四是下一步必胜的棋形。
        \item 当对手形成活三(下一步就能形成活四的三枚棋子)时,需要立刻堵上,防止对手形成活四。因此活三的进攻能力和冲四相近。
        \item 当一名玩家同时形成两个活三或冲四之后,无论对手堵哪条线,都可以把另一条线变为活四。所以双活三、双冲四、活三+冲四都是下两步必胜的棋形。
        \item 若采用$N$手交换规则,第二名玩家一定会在判断形势之后,选择更有利的一方继续下棋。所以为了防止第二名玩家得利,第一名玩家需要摆出一个尽可能均势的开局。

              在一手交换规则下,黑棋第一手下在不同位置时的胜率见右图。一个点在图中的颜色越接近绿色(代表胜率越接近$50\%$),这个点就越适合作为开局。
    \end{itemize}
\end{minipage}
\begin{minipage}[t]{.3\textwidth}
    \begin{figure}[H]
        \centering
        \begin{tikzpicture}[scale=0.2]
            \gomokuBoard
        \end{tikzpicture}
        \caption{标准的五子棋棋盘}
    \end{figure}
    \begin{figure}[H]
        \centering
        \begin{tikzpicture}[scale=0.2]
            \gomokuBoard

            \goLikeBlackStone{2}{11}
            \goLikeBlackStone{3}{11}
            \goLikeBlackStone{4}{11}
            \goLikeBlackStone{5}{11}
            \goLikeBlackStone{6}{11}

            \goLikeWhiteStone{11}{9}
            \goLikeWhiteStone{11}{10}
            \goLikeWhiteStone{11}{11}
            \goLikeWhiteStone{11}{12}
            \goLikeWhiteStone{11}{13}

            \goLikeWhiteStone{2}{2}
            \goLikeWhiteStone{3}{3}
            \goLikeWhiteStone{4}{4}
            \goLikeWhiteStone{5}{5}
            \goLikeWhiteStone{6}{6}

            \goLikeBlackStone{9}{5}
            \goLikeBlackStone{10}{4}
            \goLikeBlackStone{11}{3}
            \goLikeBlackStone{12}{2}
            \goLikeBlackStone{13}{1}
        \end{tikzpicture}
        \caption{五连的例子}
    \end{figure}
    \begin{figure}[H]
        \centering
        \begin{tikzpicture}[scale=0.2]
            \gomokuBoard

            \goLikeBlackStone{2}{11}
            \goLikeBlackStone{3}{11}
            \goLikeBlackStone{4}{11}
            \goLikeBlackStone{5}{11}

            \goLikeWhiteStone{11}{10}
            \goLikeWhiteStone{11}{11}
            \goLikeWhiteStone{11}{12}
            \goLikeWhiteStone{11}{13}

            \goLikeWhiteStone{3}{3}
            \goLikeWhiteStone{4}{4}
            \goLikeWhiteStone{5}{5}
            \goLikeWhiteStone{6}{6}

            \goLikeBlackStone{10}{4}
            \goLikeBlackStone{11}{3}
            \goLikeBlackStone{12}{2}
            \goLikeBlackStone{13}{1}
        \end{tikzpicture}
        \caption{活四的例子}
    \end{figure}
    \begin{figure}[H]
        \centering
        \begin{tikzpicture}[scale=0.2]
            \gomokuBoard

            \goLikeBlackStone{2}{11}
            \goLikeBlackStone{3}{11}
            \goLikeBlackStone{4}{11}

            \goLikeWhiteStone{11}{10}
            \goLikeWhiteStone{11}{12}
            \goLikeWhiteStone{11}{13}

            \goLikeWhiteStone{3}{3}
            \goLikeWhiteStone{4}{4}
            \goLikeWhiteStone{5}{5}

            \goLikeBlackStone{10}{4}
            \goLikeBlackStone{11}{3}
            \goLikeBlackStone{13}{1}
        \end{tikzpicture}
        \caption{活三的例子}
    \end{figure}
    \begin{figure}[H]
        \centering
        \includegraphics[width=.65\textwidth]{gomoku_firstmove.png}
        \caption{黑棋第一手下在不同位置时的胜率}
    \end{figure}
\end{minipage}

\end{document}