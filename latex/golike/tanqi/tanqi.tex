\documentclass[../../main.tex]{subfiles}
\graphicspath{{img/}}
\begin{document}

\section{弹棋}

\begin{minipage}[t]{.7\textwidth}

    \paragraph*{游戏人数} $2$人

    \paragraph*{游戏道具} 大小和围棋棋盘相近的木制棋盘,黑色和白色棋子(为避免棋子损坏,建议游戏在平坦的地面上进行)

    \paragraph*{游戏目标} 抢先把对方所有棋子都弹出棋盘

    \paragraph*{游戏过程}

    \begin{enumerate}
        \item $2$名玩家中,一名玩家下黑棋,另一名玩家下白棋。
        \item 在开始弹棋之前,双方首先需要在棋盘上布好相同数量的棋子。
        \item 棋盘分为黑棋半场和白棋半场。只能在自己的半场内布子。
        \item 布子完成之后,从下白棋的玩家开始,双方轮流弹棋子。
        \item 只能弹自己颜色的棋子。棋子可以接触任意颜色的棋子,也可以不接触任何棋子。
        \item 当对方的棋子被弹出棋盘时,被弹出棋盘的棋子放回棋盒。
        \item 特别地,如果自己有棋子被弹出棋盘,采用如下规则:
              \begin{enumerate}
                  \item 如果对方没有棋子被弹出棋盘:本回合自己被弹出棋盘的棋子可以复活。复活是指:由棋子的所有者把棋子重新放在棋盘上,但不能和已经在棋盘上的棋子重叠。
                  \item 如果对方也有棋子被弹出棋盘:本回合自己被弹出棋盘的棋子不能复活。本回合对方被弹出棋盘的棋子由对方进行复活。
              \end{enumerate}
        \item 如果轮到一名玩家弹棋时这名玩家已经失去所有棋子,那么游戏结束。棋盘上还有棋子的一方胜利。
    \end{enumerate}

    \paragraph*{基本策略}

    \begin{itemize}
        \item 进攻对方棋子是最基本、最重要的策略。
        \item 进攻对方棋子时,需要集中注意力,控制好力度和方向。争取弹出对方的棋子,同时也要避免自己的棋子被弹出。
        \item 如果没有好的进攻机会,可以尝试把自己的棋子弹至中心,或者巩固自己的阵型,争取不给对手留下进攻机会;也可以尝试破坏对手的阵型,创造进攻机会。
    \end{itemize}
\end{minipage}
\begin{minipage}[t]{.3\textwidth}
    \begin{figure}[H]
        \centering
        \begin{tikzpicture}[scale=0.2]
            \goBoard
        \end{tikzpicture}
        \caption{弹棋棋盘}
    \end{figure}
\end{minipage}

\end{document}